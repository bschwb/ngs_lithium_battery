\documentclass[a4paper,11pt]{scrartcl}
\usepackage{fullpage}

\usepackage[english]{babel}
\usepackage[utf8]{inputenc}

\usepackage[version=4]{mhchem}

\usepackage{commath}
\usepackage[retainorgcmds]{IEEEtrantools}

\newcommand*{\n}{n_{\ce{Li}}}
\newcommand*{\nv}{\hat{n}}
\newcommand*{\pn}{\frac{\partial{}\n}{\partial{}t}}
\newcommand*{\F}{\mathcal{F}}
\newcommand*{\dv}[1]{\nabla\cdot\left({#1}\right)}
\newcommand*{\I}[1]{\int_{\Omega}{#1}\dif{}x}
\newcommand*{\Ib}[1]{\int_{\partial \Omega}{#1}\dif{}x}
\newcommand*{\Dt}{\Delta t}

\begin{document}

Our model equations are:
\begin{IEEEeqnarray*}{rCl}
 \pn &=& \dv{D(\n) \nabla \n} +
  \dv{\frac{D(\n) z(x) \F \n}{R T} \nabla \phi} \\
0 &=& \frac{\partial\rho}{\partial{}t} =
  \dv{\kappa_T(x) \nabla \phi} + \dv{\frac{D(\n) z(x) \F \n}{R T} \nabla \n}
\end{IEEEeqnarray*}

\section*{Variational Formulation}

\begin{IEEEeqnarray*}{rCl}
  \I{\pn v} &=& \I{\dv{D(\n) \nabla \n} v}
  +\I{\dv{\frac{D(\n) z(x) \F \n}{RT} \nabla \phi} v} \\
&=& -\I{D(\n) \nabla \n \cdot \nabla v} + \Ib{D(\n)(\nabla\n \cdot \nv) v} \\
&&-\I{\frac{D(\n) z(x) \F \n}{R T} \nabla \phi \cdot \nabla v}
-\Ib{\left(\frac{D(\n) z(x) \F \n}{R T} \nabla \phi \cdot \nv\right)  v}
\end{IEEEeqnarray*}

\begin{IEEEeqnarray*}{rCl}
0 &=& - \I{\kappa_T(x) \nabla \phi \cdot \nabla \psi} +
\Ib{(\kappa_T(x) \nabla \phi \cdot \nv) \psi} \\
&& - \I{\frac{D(\n) z(x) \F \n}{R T} \nabla \n \cdot \nabla \psi} +
\Ib{\left(\frac{D(\n) z(x) \F \n}{R T} \nabla \n \cdot \hat{n}\right) \psi}
\end{IEEEeqnarray*}

$\sigma E = J$

\section*{Time Stepping}
We write our system in a shorter form:
\begin{IEEEeqnarray*}{rCl}
n_t &=& f(n) + g(n) \phi \\
0  &=& A \phi + h(n)
\end{IEEEeqnarray*}
We first try the Crank-Nicolson method, because it is unconditionally stable.
Another joice would be the implicit Euler method.
\begin{IEEEeqnarray*}{rCl}
\frac{n(t+\Dt)-n(t)}{\Dt} &=& \frac{1}{2}(f(n(t+\Dt)) + g(n(t+\Dt))\phi(t+\Dt) +
f(n(t))+g(n(t))\phi(t)) \\
0 &=& \frac{1}{2}\left(A \phi(t+\Dt) + h(n(t+\Dt)) + A\phi(t) + h(n(t))\right)
\end{IEEEeqnarray*}

One way to solve the system would be to eliminate $\phi(t+\Dt)$ by
rearranging the second equation
  \[\phi(t+\Dt) = - A^{-1} h(n(t+\Dt)) - \phi(t) - h(n(t))\]
and then plugging that into the first equation.

\section*{Questions}
\begin{enumerate}
\item Activity coefficient?

  There's a difference between molal and molar coefficient! Which one is applicable?
  See~\cite{doyle95} chapter 4.
  They also used idealized versions, because measurement is hard.
  Also in~\cite{garcia2005} they mention, that for pure and dilute compositions
  the activity coefficient satisfies Raoult's and Henry's law.

\item How to incorporate boundary conditions into time stepping?

  They are part of the (non-)linear functions $f, g, A, h$.

\end{enumerate}

\bibliography{model}{}
\bibliographystyle{alpha}


\end{document}