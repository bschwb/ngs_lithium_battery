\documentclass[a4paper]{article}

\usepackage{cite}
\usepackage{url}
\usepackage{hyperref}

%% Language and font encodings
\usepackage[english]{babel}
\usepackage[utf8]{inputenc}

\usepackage{booktabs}
% \usepackage{tabu}
% \usepackage[T1]{fontenc}

\usepackage[version=4]{mhchem}

%% Sets page size and margins
\usepackage[a4paper,top=3cm,bottom=2cm,left=3cm,right=3cm,marginparwidth=1.75cm]{geometry}

%% Useful packages
\usepackage{amsmath}
\usepackage{graphicx}
% \usepackage[colorlinks=true, allcolors=blue]{hyperref}

\title{Project Proposal for SF2567}
\author{Bernd Schwarzenbacher}
\date{23. January 2018}

\begin{document}
\maketitle

\section*{Summary of the Proposal}

I propose a project in which I consider a multiphysics 2D model for Lithium-Ion Batteries as
described in \cite{Garcia2005}.
The model will be implemented and solved via the Finite Element Method in the
software Netgen/NGSolve \cite{Netgen}\cite{NGSolve}.
As in \cite{Garcia2005} I will validate the simulation results with the same
experimental data of a \ce{Li_yC6|Li_xMn2O4} battery\cite{Doyle1995}.

\section*{Background}

Lithium-Ion batteries are widely used today as in for example consumer electronics and electrical vehicles.
To study the behavior of those batteries under different circumstances a wide
array of models has been developed.
Often those models are then solved by numerical models when analytical methods
are not applicable.

\noindent The core parts of the model under investigation are:
\begin{itemize}
  \item kinetic equations for the single lithium ions and electrons
  \item electrode - electrolyte interface kinetics
  \item stresses arrising due to volume changes in the electrodes during the charge/discharge cycles
\end{itemize}

\section*{Project Goal and Objectives}

The goal of this project is to implement the model described in
\cite{Garcia2005} and compare the results with experimental data.

The first step will be to further study the paper and references to fill gaps in physical and chemical understanding, as well as fully understand the model.
Afterwards the model will be programmatically described in NGSolve
\cite{NGSolve} via the Python interface and a spacial discretization by suitable
finite elements as well as time integration will be chosen and implemented.
In the end the simulation results will be compared to experimental data for validation.

\bibliographystyle{plain}
\bibliography{refs}

\end{document}
              